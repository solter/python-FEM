\documentclass[10pt,letterpaper,fleqn]{article}
% Included at the end of the template are examples of figures and tables
%
% All images you wish to include must be in .png format.

% Load packages.  All header information is declared in PH3480.sty.
% Useful end user packages include:
% 	graphicx
%	subfig
% 	url
% 	psfrag
%	amsmath
%	amsfonts
%	color
% For a full list, please view PH3480.sty.
% Place additional package inclusions immediately proceeding this line.

% Load packages
\usepackage{PH3480}

% Graphics info
\graphicspath{{../out/}}
\DeclareGraphicsExtensions{.png}

% Footnote format
% \symbolfootnote[#]{text...}
% 1 - *
% 2 - dagger
% 3 - double dagger

% --- SETUP PAGE HEADER --- %
\pagenumbering{arabic}
\pagestyle{fancy}
\fancyhf{}
\fancyhead[L]{\textsl{MA 5629: NUM. PDE's}}

% You should only need to change the following line in this section
\fancyhead[R]{\textsl{Final Proj. Report} - Page \thepage}

% --- SETUP PAGE HEADER --- %

%%%%%%%%%%%%%%%%%%%%%%%%%%%%%%%%%%%%%%%%%%%%%%%%%%%%%%%%%%%%%%%%%%%%%%%%%%%%%%%%
%%%%%%%%%%%%%%%%%%%%%%%%%%%%% DOCUMENT START %%%%%%%%%%%%%%%%%%%%%%%%%%%%%%%%%%%
%%%%%%%%%%%%%%%%%%%%%%%%%%%%%%%%%%%%%%%%%%%%%%%%%%%%%%%%%%%%%%%%%%%%%%%%%%%%%%%%
\begin{document}
\doublespacing

% <--- TITLE START ---> %
\begin{flushright}
Numerical PDE's final Project\\
Peter Solfest
Code available at \url{https://github.com/solter/python-FEM}
\end{flushright}
% <---- TITLE END ----> %


% <--- Problem 1 ---> %
\section{Problem 1}
The equation $u_t + \left(\dfrac{u^2}{2}\right)_x = \varepsilon u_{xx}$ was solved on the interval $x \in [-1,1]$, $\varepsilon = 0.01$ until $T = 0.1$.

With the boundary conditions $u(-1) = 1$ and $u(1) = 0$, and initial conditions $u(x,0) = -.5x + .5$ and $u(x,0) = 1 - x^2$.

$[-1,1]$ was meshed into $N$ equi-length intervals with $N = 40,80$.

Both a standard FEM method and a streamline diffusion method were used
to solve the system, using forward euler for the time integration,
with 1 and 3 point quadrature schemes used for the nonlinear term.

The figures in appendix \ref{app:FEM} display the solutions.

% <--- Problem 1 ---> %
\section{Problem 2}
The equation $u_t + \left(\dfrac{u^2}{2}\right)_x = 0$ was solved on the interval $x \in [-1,1]$, until $T = 0.05, 0.1$ and $0.2$.

A periodic boundary condition was imposed, with the initial condition
$u(x,0) = 0.5 (1 + \sin(\pi t))$.

$[-1,1]$ was meshed into $160$ equi-length intervals.

Finite volume methods were used to solve this.
An ENO scheme (both $3^{rd}$ and $1^{st}$ order) were used for
the interface value reconstructions.
Both forward euler and a TVD RK3 solver were used for the time integration,
and the numerical fluxes were reconstructed using Godunov and Global Lax-Friedrichs schemes.

The figures in appendix \ref{app:FVM} display the solutions.


% <---- NORMAL MODE END ----> %
\appendix
\section{Problem 1 Figures}\label{app:FEM}
Note that above each figure represents different times during the solution.
The figures are labelled via the number of intervals ($N$),
whether a standard or streamline method was used, and the initial condition.

\begin{figure}[h!]
        \centering
        \begin{subfigure}[b]{0.4\textwidth}
                \includegraphics[width=\textwidth]{fin401A_1}
        \end{subfigure}%
        ~ 
        \begin{subfigure}[b]{0.4\textwidth}
                \includegraphics[width=\textwidth]{fin401A_2}
        \end{subfigure}
        
        \begin{subfigure}[b]{0.4\textwidth}
                \includegraphics[width=\textwidth]{fin401A_3}
        \end{subfigure}
        ~
        \begin{subfigure}[b]{0.4\textwidth}
                \includegraphics[width=\textwidth]{fin401A_4}
        \end{subfigure}
        \caption{$N = 40$ via standard with $u(x,0) = -.5x + .5$}
\end{figure}

\begin{figure}[h!]
        \centering
        \begin{subfigure}[b]{0.4\textwidth}
                \includegraphics[width=\textwidth]{fin401B_1}
        \end{subfigure}%
        ~ 
        \begin{subfigure}[b]{0.4\textwidth}
                \includegraphics[width=\textwidth]{fin401B_2}
        \end{subfigure}
        
        \begin{subfigure}[b]{0.4\textwidth}
                \includegraphics[width=\textwidth]{fin401B_3}
        \end{subfigure}
        ~
        \begin{subfigure}[b]{0.4\textwidth}
                \includegraphics[width=\textwidth]{fin401B_4}
        \end{subfigure}
        \caption{$N = 40$ via streamline with $u(x,0) = -.5x + .5$}
\end{figure}

\begin{figure}[h!]
        \centering
        \begin{subfigure}[b]{0.4\textwidth}
                \includegraphics[width=\textwidth]{fin801A_1}
        \end{subfigure}%
        ~ 
        \begin{subfigure}[b]{0.4\textwidth}
                \includegraphics[width=\textwidth]{fin801A_2}
        \end{subfigure}
        
        \begin{subfigure}[b]{0.4\textwidth}
                \includegraphics[width=\textwidth]{fin801A_3}
        \end{subfigure}
        ~
        \begin{subfigure}[b]{0.4\textwidth}
                \includegraphics[width=\textwidth]{fin801A_4}
        \end{subfigure}
        \caption{$N = 80$ via standard with $u(x,0) = -.5x + .5$}
\end{figure}

\begin{figure}[h!]
        \centering
        \begin{subfigure}[b]{0.4\textwidth}
                \includegraphics[width=\textwidth]{fin801B_1}
        \end{subfigure}%
        ~ 
        \begin{subfigure}[b]{0.4\textwidth}
                \includegraphics[width=\textwidth]{fin801B_2}
        \end{subfigure}
        
        \begin{subfigure}[b]{0.4\textwidth}
                \includegraphics[width=\textwidth]{fin801B_3}
        \end{subfigure}
        ~
        \begin{subfigure}[b]{0.4\textwidth}
                \includegraphics[width=\textwidth]{fin801B_4}
        \end{subfigure}
        \caption{$N = 80$ via standard with $u(x,0) = -.5x + .5$}
\end{figure}

\begin{figure}[h!]
        \centering
        \begin{subfigure}[b]{0.4\textwidth}
                \includegraphics[width=\textwidth]{fin801A2_1}
        \end{subfigure}%
        ~ 
        \begin{subfigure}[b]{0.4\textwidth}
                \includegraphics[width=\textwidth]{fin801A2_2}
        \end{subfigure}
        
        \begin{subfigure}[b]{0.4\textwidth}
                \includegraphics[width=\textwidth]{fin801A2_3}
        \end{subfigure}
        ~
        \begin{subfigure}[b]{0.4\textwidth}
                \includegraphics[width=\textwidth]{fin801A2_4}
        \end{subfigure}
        \caption{$N = 80$ via standard with $u(x,0) = 1 - x^2$}
\end{figure}

\begin{figure}[h!]
        \centering
        \begin{subfigure}[b]{0.4\textwidth}
                \includegraphics[width=\textwidth]{fin801B2_1}
        \end{subfigure}%
        ~ 
        \begin{subfigure}[b]{0.4\textwidth}
                \includegraphics[width=\textwidth]{fin801B2_2}
        \end{subfigure}
        
        \begin{subfigure}[b]{0.4\textwidth}
                \includegraphics[width=\textwidth]{fin801B2_3}
        \end{subfigure}
        ~
        \begin{subfigure}[b]{0.4\textwidth}
                \includegraphics[width=\textwidth]{fin801B2_4}
        \end{subfigure}
        \caption{$N = 80$ via standard with $u(x,0) = 1 - x^2$}
\end{figure}

\section{Problem 2 Figures}\label{app:FVM}
Note that above each figure represents different times during the solution.
Beneath each group of 4 figures the following code is used
\begin{itemize}
	\item RO: The ENO reconstruction accuracy used
	\item OT: The order of accuracy for the time integration method (1 is forward euler, 3 is TVD RK3)
	\item flx: Whether a Godunov or GLF flux was used 
\end{itemize}

\begin{figure}[h!]
        \centering
        \begin{subfigure}[b]{0.4\textwidth}
                \includegraphics[width=\textwidth]{2A1_1}
        \end{subfigure}%
        ~ 
        \begin{subfigure}[b]{0.4\textwidth}
                \includegraphics[width=\textwidth]{2A1_2}
        \end{subfigure}
        
        \begin{subfigure}[b]{0.4\textwidth}
                \includegraphics[width=\textwidth]{2A1_3}
        \end{subfigure}
        ~
        \begin{subfigure}[b]{0.4\textwidth}
                \includegraphics[width=\textwidth]{2A1_4}
        \end{subfigure}
        \caption{RO = 1, OT = 1, Godunov}
\end{figure}

\begin{figure}[h!]
        \centering
        \begin{subfigure}[b]{0.4\textwidth}
                \includegraphics[width=\textwidth]{2A2_1}
        \end{subfigure}%
        ~ 
        \begin{subfigure}[b]{0.4\textwidth}
                \includegraphics[width=\textwidth]{2A2_2}
        \end{subfigure}
        
        \begin{subfigure}[b]{0.4\textwidth}
                \includegraphics[width=\textwidth]{2A2_3}
        \end{subfigure}
        ~
        \begin{subfigure}[b]{0.4\textwidth}
                \includegraphics[width=\textwidth]{2A2_4}
        \end{subfigure}
        \caption{RO = 1, OT = 1, GLF}
\end{figure}

\begin{figure}[h!]
        \centering
        \begin{subfigure}[b]{0.4\textwidth}
                \includegraphics[width=\textwidth]{2B1_1}
        \end{subfigure}%
        ~ 
        \begin{subfigure}[b]{0.4\textwidth}
                \includegraphics[width=\textwidth]{2B1_2}
        \end{subfigure}
        
        \begin{subfigure}[b]{0.4\textwidth}
                \includegraphics[width=\textwidth]{2B1_3}
        \end{subfigure}
        ~
        \begin{subfigure}[b]{0.4\textwidth}
                \includegraphics[width=\textwidth]{2B1_4}
        \end{subfigure}
        \caption{RO = 3, OT = 3, Godunov}
\end{figure}

\begin{figure}[h!]
        \centering
        \begin{subfigure}[b]{0.4\textwidth}
                \includegraphics[width=\textwidth]{2B2_1}
        \end{subfigure}%
        ~ 
        \begin{subfigure}[b]{0.4\textwidth}
                \includegraphics[width=\textwidth]{2B2_2}
        \end{subfigure}
        
        \begin{subfigure}[b]{0.4\textwidth}
                \includegraphics[width=\textwidth]{2B2_3}
        \end{subfigure}
        ~
        \begin{subfigure}[b]{0.4\textwidth}
                \includegraphics[width=\textwidth]{2B2_4}
        \end{subfigure}
        \caption{RO = 3, OT = 3, GLF}
\end{figure}

\begin{figure}[h!]
        \centering
        \begin{subfigure}[b]{0.4\textwidth}
                \includegraphics[width=\textwidth]{21_1}
        \end{subfigure}%
        ~ 
        \begin{subfigure}[b]{0.4\textwidth}
                \includegraphics[width=\textwidth]{21_2}
        \end{subfigure}
        
        \begin{subfigure}[b]{0.4\textwidth}
                \includegraphics[width=\textwidth]{21_3}
        \end{subfigure}
        ~
        \begin{subfigure}[b]{0.4\textwidth}
                \includegraphics[width=\textwidth]{21_4}
        \end{subfigure}
        \caption{RO = 3, OT = 1, Godunov}
\end{figure}

\begin{figure}[h!]
        \centering
        \begin{subfigure}[b]{0.4\textwidth}
                \includegraphics[width=\textwidth]{22_1}
        \end{subfigure}%
        ~ 
        \begin{subfigure}[b]{0.4\textwidth}
                \includegraphics[width=\textwidth]{22_2}
        \end{subfigure}
        
        \begin{subfigure}[b]{0.4\textwidth}
                \includegraphics[width=\textwidth]{22_3}
        \end{subfigure}
        ~
        \begin{subfigure}[b]{0.4\textwidth}
                \includegraphics[width=\textwidth]{22_4}
        \end{subfigure}
        \caption{RO = 1, OT = 3, Godunov}
\end{figure}

\begin{figure}[h!]
        \centering
        \begin{subfigure}[b]{0.4\textwidth}
                \includegraphics[width=\textwidth]{2B2_t100_1}
        \end{subfigure}%
        ~ 
        \begin{subfigure}[b]{0.4\textwidth}
                \includegraphics[width=\textwidth]{2B2_t100_2}
        \end{subfigure}
        
        \begin{subfigure}[b]{0.4\textwidth}
                \includegraphics[width=\textwidth]{2B2_t100_3}
        \end{subfigure}
        ~
        \begin{subfigure}[b]{0.4\textwidth}
                \includegraphics[width=\textwidth]{2B2_t100_4}
        \end{subfigure}
        \caption{RO = 1, OT = 1, Godunov}
\end{figure}

\begin{figure}[h!]
        \centering
        \begin{subfigure}[b]{0.4\textwidth}
                \includegraphics[width=\textwidth]{2B_t100_1}
        \end{subfigure}%
        ~ 
        \begin{subfigure}[b]{0.4\textwidth}
                \includegraphics[width=\textwidth]{2B_t100_2}
        \end{subfigure}
        
        \begin{subfigure}[b]{0.4\textwidth}
                \includegraphics[width=\textwidth]{2B_t100_3}
        \end{subfigure}
        ~
        \begin{subfigure}[b]{0.4\textwidth}
                \includegraphics[width=\textwidth]{2B_t100_4}
        \end{subfigure}
        \caption{RO = 1, OT = 1, GLF}
\end{figure}
% <--- REFERENCES START ---> %
\pagebreak
\bibliographystyle{unsrt}
\bibliography{references}
\thispagestyle{fancy}
% <---- REFERENCES END ----> %
%%%%%%%%%%%%%%%%%%%%%%%%%%%%%%%%%%%%%%%%%%%%%%%%%%%%%%%%%%%%%%%%%%%%%%%%%%%%%%%%
%%%%%%%%%%%%%%%%%%%%%%%%%%%%%% DOCUMENT END  %%%%%%%%%%%%%%%%%%%%%%%%%%%%%%%%%%% 
%%%%%%%%%%%%%%%%%%%%%%%%%%%%%%%%%%%%%%%%%%%%%%%%%%%%%%%%%%%%%%%%%%%%%%%%%%%%%%%%
\end{document}
  
